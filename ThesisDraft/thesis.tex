\documentclass[a4paper, 12pt]{book}
\usepackage{amssymb}
\usepackage{xspace}
\usepackage{tikz}
\usepackage[newitem,newenum,neverdecrease]{paralist}
\usepackage{amsmath,amssymb,amsfonts,mathrsfs,latexsym,stmaryrd}
\usepackage{mathtools}
\usepackage{amsthm}
\usepackage{ifpdf}
\usepackage{cases}
\usepackage{ragged2e} 	% https://tex.stackexchange.com/questions/89680/how-can-one-set-full-justification-within-left-justified-raggedright-text
\usepackage{breakcites} % https://tex.stackexchange.com/questions/2773/how-do-i-make-latex-push-long-citations-to-a-new-line
\usepackage{hyperref}	% http://tex.stackexchange.com/questions/73862/how-can-i-make-a-clickable-table-of-contents
\usepackage{fancyhdr}	% https://www.sharelatex.com/blog/2013/08/06/thesis-series-pt2.html
\usepackage{titlesec} 	% http://tex.stackexchange.com/questions/11444/how-to-format-the-chapter-heading
\usepackage[T1]{fontenc}
\usepackage[utf8]{inputenc}
%\usepackage{setspace}
\usepackage{xcolor}
\usepackage{geometry}
\usepackage{layout}		% http://tex.stackexchange.com/questions/50258/margins-of-book-class
\input{fixpseudocode}	% for pseudocodes

%%% Formattazione del header del Capitolo
\titleformat
{\chapter} % command
[display] % shape
{\bfseries\Huge\itshape} % format
{\flushright \color{black!45}{\Large Chapter \thechapter}} % label
{0.5ex} % sep
{
    \rule{\textwidth}{3pt}
    \vspace{1ex}
    \centering
} % before-code
[
\vspace{-0.5ex}%
\rule{\textwidth}{3pt}
] % after-code
%%% end

% https://tex.stackexchange.com/questions/95488/list-of-figures-and-page-numbering
\makeatletter
\newcommand{\emptypage}[1]{%
  \cleardoublepage
  \begingroup
  \let\ps@plain\ps@empty
  \pagestyle{empty}
  #1
  \cleardoublepage}
\makeatletter

% base line strech (default 1.0) -- interlinea
\renewcommand{\baselinestretch}{1.2}

\begin{document}
%%% Nuova geometria per la pagina del titolo
% visto che la classe del documento è book, le pagine pari e dispari avranno geometrie diverse,
% mentre quella del titolo deve essere unica!
\newgeometry{
  top=2cm,
  bottom=2.5cm,
  left=2.5cm,
  right=2.5cm,
  headsep=25pt,
  headheight=14.5pt
}

%%% Pagina del titolo
\begin{titlepage}
%	\topskip0pt
	%\vspace*{\fill}
	\centering
%	\vspace*{20mm}
	\includegraphics[]{logo.png}\\
	\vspace*{1cm}
	\huge \textbf{\textsc{Università degli Studi di Ferrara}}
	\Large \textsc{Corso di Laurea in Informatica}
	
	\vspace*{1.5cm}
	\hrule width \hsize \kern 1mm \hrule width \hsize height 2pt
	\vspace*{10mm}
	\Huge \emph{\textbf{Data diodes for secure Industry 4.0 networking: \\ The railway track monitoring use case}}
	\vspace*{10mm}
	\hrule width \hsize height 2pt
	\vspace*{1mm}
	\hrule width \hsize \kern 1mm
	
	\vspace*{5mm}
	\begin{minipage}{0.46\textwidth}
		\begin{flushleft} \Large
			\emph{Relatore:}\\
			\Large \textbf{Prof. Carlo \textsc{Giannelli}}
            \emph{Tutor Aziendale:}\\
			\Large \textbf{Ing. Roberto \textsc{Giansante}}
            \emph{Co-Relatore:}\\
			\Large \textbf{Dr. Giulio \textsc{Riberto}}
		\end{flushleft}
	\end{minipage}
	\begin{minipage}{0.45\textwidth}
		\begin{flushright} \Large
			\emph{Laureando:} \\
			\Large \textbf{Giuseppe \textsc{Cervone}}
		\end{flushright}
	\end{minipage}
	
	\vspace*{20mm}
	\Large \textsc{Anno Accademico $2020-2021$}
\end{titlepage}
\restoregeometry

\addtocontents{toc}{~\hfill\textbf{Page}\par}	% https://texblog.org/2011/09/09/10-ways-to-customize-tocloflot/
\pagestyle{empty}
\clearpage
\tableofcontents
\thispagestyle{empty}
\addtocontents{toc}{\protect\thispagestyle{empty}}	% http://tex.stackexchange.com/questions/2995/removing-page-number-from-toc

\chapter{Ringraziamenti}


\chapter{Riassunto}

Qui, inserire un breve riassunto del lavoro in italiano.


\chapter{Introduction}
%%% Fancy header settings, queste impostazioni vanno fatte solo una volta all'inizio del primo capitolo!
\pagestyle{fancy}
\fancyhf{}
\renewcommand{\headrulewidth}{2pt}
\fancyhead[EL]{\textbf{\textsf{\nouppercase\thepage}}}
\fancyhead[ER]{\textbf{\textsf{\nouppercase\leftmark}}}
\fancyhead[OR]{\textbf{\textsf{\nouppercase\thepage}}}
\fancyhead[OL]{\textbf{\textsf{\nouppercase {\rightmark}}}}
%%% end

%%% all'inizio di ogni capitolo, questa impostazione rimuove il numero di pagina, provare a commentare per vedere la differenza
\thispagestyle{empty}

asd
\newpage
Qua inizia la numerazione delle pagine, guardare in modificaalto della pagina

\chapter{Industry 4.0 and cybersecurity}
Modern history is often categorized taking 
\section{Theory behind the Industry 4.0 movement}
\section{Typical use cases}
\section{Industrial IoT and the cybersecurity problem}

\chapter{Alstom group and their problem}

\section{Dissecting Alstom as a company}
\section{Alstom solution for railway track monitoring}
focusing on their shift to computer science oriented solutions

\chapter{Project}

\section{Alstom issues with interacting with machines under private networks}
\section{Goals of the thesis}
\section{Top-down description of the implemented solution}

\chapter{Conclusion}

\bibliographystyle{plain}

\end{document} 
